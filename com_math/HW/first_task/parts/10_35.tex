\subsubsection*{10.35}
Показатель преломления определяется соотношением
\begin{equation*}
	n = \sqrt{1 + 4 \pi \alpha N}
\end{equation*}
где $\alpha$ --- поляризуемость молекул газа (в гауссовской системе), а $N$ --- их концентрация. Принимая во внимание, что 
\begin{equation*}
	N(h) = \frac{P_0}{kT}\exp \left( -\frac{mg_Bh}{kT}\right) 
\end{equation*}
где $P_0 / kT$ --- концентрация молекул при $h = 0$, получим
\begin{equation*}
	n = \sqrt{1+ 4\pi \alpha \frac{P_0}{kT} \exp \left( -\frac{mg_Bh}{kT}\right) } \approx 1 + 2\pi \alpha \frac{P_0}{kT} \left( 1 - \frac{mg_Bh}{kT}\right) 
\end{equation*}
Радиус кривизны луча, пущенного горизонтального вблизи поверхности планеты, есть
\begin{equation*}
	r \approx - \frac{(kT)^2}{2\pi \alpha P_0 m g_B}
\end{equation*}
\begin{equation*}
	n - 1 \approx (n_0 - 1) \left[1 - mgh/(kT)\right]
\end{equation*}
Т.к. $(n-1) << 1$
\begin{equation*}
	r = -kT \left[(n_0 - 1)mg\right]
\end{equation*}
Для Земли $n_0 = 1,0003$ откуда $r \approx -2,9 \cdot 10^4 Км$. Так как радиус Земли равен $6,4 \cdot 10^3 Км$ и $n_0 -1 \sim p_0$, для круговой рефракции давление (и плотность) в атмосфере Земли должны быть увеличены в $4,5$ раза.