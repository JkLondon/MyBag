\subsubsection*{I.6.4}
У нас есть 2 ряда:
$$\ln (1+x) = x - \frac{x^2}{2} + \frac{x^3}{3} - \frac{x^4}{4} + \dots + (-1)^{k+1} \frac{x^k}{k}$$
$$\ln (1+x) = m\ln 2 - 2 \left( y + \frac{y^3}{3} + \dots +  \frac{y^{2k-1}}{2k - 1} + \dots \right)$$
где $1 + x = 2^m z$ и $z \in \left[0,5; 1\right]$, а $y = \frac{1 - z}{1 + z}$\\
Второй ряд имеет преимущества и недостатки. Во-первых: в таком случае у нас $y \in \left[0; \frac{1}{3}\right]$, таким образом, каким бы большим у нас не было число $x$, мы можем претендовать на необходимую погрешность, отбрасывая гораздо меньшее количество членов (Замечу, что в 1ом ряду у нас напротив при больших $x$ всё плохо с погрешностью). Но, сразу же нужно заметить, что при $y \rightarrow 0$ становится тяжелее вычислить члены этого ряда, что затрудняет оценку точности (в таком случае у нас $1+x \rightarrow 2^m$). Особенно будет заметно преимущество первого ряда при $x \rightarrow -0$, где погрешность у второго будет легче вычисляться за счет существования квадратного члена при малых $x$. Погрешность в рядах тейлора всегда будет не более последнего отбрасываемого члена (из теории)