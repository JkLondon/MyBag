\documentclass[a4paper,12pt]{article} % добавить leqno в [] для нумерации слева

%%% Работа с русским языком
\usepackage{cmap}					% поиск в PDF
\usepackage{mathtext} 				% русские буквы в фомулах
\usepackage[T2A]{fontenc}			% кодировка
\usepackage[utf8]{inputenc}			% кодировка исходного текста
\usepackage[english,russian]{babel}	% локализация и переносы
\usepackage{float}
\usepackage{graphicx}
%%% Дополнительная работа с математикой
\usepackage{amsmath,amsfonts,amssymb,amsthm,mathtools} % AMS
\usepackage{icomma} % "Умная" запятая: $0,2$ --- число, $0, 2$ --- перечисление

%% Номера формул
%\mathtoolsset{showonlyrefs=true} % Показывать номера только у тех формул, на которые есть \eqref{} в тексте.

%% Шрифты
\usepackage{euscript}	 % Шрифт Евклид
\usepackage{mathrsfs} % Красивый матшрифт
\newtheorem{theorem}{Theorem}
\newtheorem{lemma}[theorem]{Lemma}
\theoremstyle{definition}
\newtheorem{definition}{Definition}[section]
%% Свои команды
\DeclareMathOperator{\sgn}{\mathop{sgn}}

%% Перенос знаков в формулах (по Львовскому)
\newcommand*{\hm}[1]{#1\nobreak\discretionary{}
	{\hbox{$\mathsurround=0pt #1$}}{}}

%%% Заголовок
\author{Илья Михеев}
\title{Билеты по Матану, прости Господи}
\date{last upd \today  }

\begin{document} % конец преамбулы, начало документа
	
\maketitle 
	
\part{Свёртки и приближение функций бесконено гладкими}
\section{Свёртка функций и её асоциативность. Дифференцирование свёртки}
\subsection{Определение}
Свёрткой функции $h(x)$ назовем такой интеграл:
\begin{equation}
	h(x) = \int_{\mathbb{R}^n} f(x - t) g(t) \, dt = \int_{\mathbb{R}^n} f(t) g(x - t) \, dt 
\end{equation}
или $h= f * g$. 
\subsection{Немного о существовании интеграла}
\begin{theorem}\label{th1} 
	Если функции $f$ и $g$ имеют конечные интегралы, то $f *g$ определена почти всюду и выполняется неравенство
	\begin{equation}
		 \int_{\mathbb{R}^n} |f * g| \, dx  \leq  \int_{\mathbb{R}^n} |f| \, dx  \cdot  \int_{\mathbb{R}^n} |g|\, dx 
	\end{equation}
	и равенство
	\begin{equation}
		 \int_{\mathbb{R}^n} f  * g \, dx  =  \int_{\mathbb{R}^n} f \, dx \cdot  \int_{\mathbb{R}^n} g \, dx 
	\end{equation}
\end{theorem}
\begin{proof}
	Функция $f(y)g(x)$ измерима по Лебегу и интеграл ее модуля равен произведению интегралов модулей $f$ и $g$ по теореме Фубини. Тогда выражение 
	\begin{equation}
		 \int_{\mathbb{R}^n \times \mathbb{R}^n} |f(x - t) g(t) \, dt dx
	\end{equation}
	также равно произведению модулей $f$ и $g$, так как различается от $|f(y) g(x)|$ линейной заменой с ед. детерминантом. Отсюда можно понять, что интегралы в неравенстве
	\begin{equation}
		\left| \int_{\mathbb{R}^n} f(x - t) g(t) \, dt \, \right| \leq  \int_{\mathbb{R}^n} |f(x - t) g(t)| \, dt 
	\end{equation}
	определены почти для всех $x$ и требуемое неравенство получается из интегрирования по $x$. Последнее равенство получается из теоремы Фубини линейной заменой $x - t  = y$.
\end{proof}
\subsection{Ассоциативность}
\begin{theorem}
	Свёртка ассоциативна, то есть:
	\begin{equation}
		f * (g * h) =  (f * g) * h
	\end{equation}
\end{theorem}
\begin{proof}
	\begin{equation}
		f * (g * h) = f * \int_{\mathbb{R}^n} g(x - t) h(t) \, dt = f * k = \int_{\mathbb{R}^n} f(x - u) \int_{\mathbb{R}^n} g(u - v) h(v) \, dv  \, du
	\end{equation}
	\begin{equation}
	(f * g) * h = \int_{\mathbb{R}^n} f(x - t) g(t) \, dt * h  = k * h = \int_{\mathbb{R}^n} \int_{\mathbb{R}^n} f(x - u - v) g(v) h(u)  \, dv  \, du
	\end{equation}
	Становится понятно, что первое равно второму после замены $s = u + v$ во втором равенстве. Также надо в верхнем переставить второй интеграл в начало (имеем право). Ну сами попробуйте короче.
\end{proof}
\subsection{Дифференцирование свёртки}
\begin{theorem}
	Если в свёртке функция $g$ интегрируема с конечным интегралом, а $f$ ограничена, также как и ее частная производная $\frac{\partial f}{\partial x_i}$. Тогда можем дифференцировать под знаком интеграла (по теореме из 2ого сема, которая имеет буквально те условия, что описаны выше)
	\begin{equation}
		\frac{\partial (f * g)}{\partial x_i} = \int_{\mathbb{R}^n} \frac{\partial f (x - t)}{\partial x_i} g(t) \, dt = \frac{\partial f}{\partial x_i} * g	
	\end{equation}
\end{theorem}
\begin{proof}
	Следует из теоремы, доказанной ранее (прошлый семестр), не уверен, что ее требуется передоказывать.
\end{proof}
\section{Бесконечно гладкие функции с компактным носителем, примеры}
Давайте для начала посмотрим на некоторую бесконечно гладкую функцию
\begin{equation}
	f(x) = 
	\begin{cases}
		0,  & \text{$x \leq 0$;} \\
		e^{-1/x},  & \text{$x > 0$.}
	\end{cases}
\end{equation}
Она бесконечно дифференцируема везде, кроме мб точки $0$. Всякая производная справа от нуля у функции имеет вид $P(1/x) e^(-1/x)$, где $P$ --- многочлен. Отсюда следует, что предел ее производной в нуле справа равен нулю. Также имеет место (Лопиталь)
\begin{equation}
	f^{(n+1)}(0) = \lim\limits_{x \rightarrow 0} \frac{f^{(n)}(x) - f^{(n)}(0)}{x} = \frac{f^{(n+1)}(x)}{1} = 0
\end{equation}
Поэтому функция $f$ бесконечно дифференцируема (бесконечно гладкая) на всей прямой. Тогда введем функцию $\varphi(x)$
\begin{equation}
	\varphi(x) = f(x + 1)f(x - 1)
\end{equation}
Которая будет бесконечно гладкой на всей прямой и будет отлична от нуля только на интервале $(-1, 1)$, на котором она будет положительна.
\begin{lemma}
	Для всякого $\varepsilon > 0$ существует бесконечно гладкая функция $\varphi_{\varepsilon} : \mathbb{R}^n \rightarrow \mathbb{R}^+$, отличная от нуля только в $U_{\varepsilon}(0)$ и такая, что 
	\begin{equation}
		\int_{\mathbb{R}^n} \varphi_{\varepsilon} (x) \, dx = 1
	\end{equation}
	Для всяких $\varepsilon > \delta > 0$ существует бесконечно гладкая функция $\psi_{\varepsilon, \delta} : \mathbb{R}^n \rightarrow [0, 1]$, отличная от  нуля только в $U_{\varepsilon}(0)$ тождественно равная 1 в $U_{\delta}(0)$.
 \end{lemma}
\begin{proof}
	В первом случае пойдет функция вида
	\begin{equation}
		\varphi_{\varepsilon}(x_1, \dots, x_n) = A \varphi(\frac{\sqrt{n} x_1}{\varepsilon}) \dots  \varphi(\frac{\sqrt{n} x_n}{\varepsilon}) 
	\end{equation}
	для уже известной функции $\varphi$ и некоторой константы $A$. Способ построения функции указывает, что в пределе одного аргумента функция ненулевая при $|x_i| \leq \frac{\varepsilon}{\sqrt{n}}$.
\end{proof}
Во втором случае сначала рассмотрим функцию одной переменной
\begin{equation}
	\psi (x) = B \int\limits_{-\infty}^{x} \varphi (t) \, dt,
\end{equation}
Где константу выбираем так, чтобы $\psi (x) \equiv 0$ при $x \leq  -1$ и $\psi (x) \equiv 1$ при $x \geq  1$. Тогда достаточно положить 
\begin{equation}
	\psi_{\varepsilon, \delta} (x) = \psi \left( \frac{\delta + \varepsilon - |x|}{\varepsilon - \delta}\right)
\end{equation}
Такая вот прикольная псевдо-ступенька.
\section{Приближение функций в $\mathbb{R}^n$ (вместе с производными) бесконечно гладкими функциями}
\subsection{Простое приближение}
\begin{theorem}
	Пусть $\varphi : \mathbb{R}^n \rightarrow R$ --- неотрицательная бесконечно гладкая функция, отличная от нуля только при $|x| \leq 1$ и пусть  $\int_{\mathbb{R}^n} \varphi(x) \, dx = 1$. Положим 
	\begin{equation}
		\varphi_k (x) = k^n \varphi (kx),
	\end{equation}
	эти функции тоже имеют единичные интегралы и $\varphi_k$ отлична от нуля только при $|x| \leq 1/k$. (Попробуйте эту лабуду представить сначала без $k^n$, а потом поймите зачем $k^n$ нужно). Теперь для непрерывной $f : \mathbb{R}^n \rightarrow \mathbb{R}$ определим свёртки
	\begin{equation}
		f_k (x) = \int_{\mathbb{R}^n} f(x - t) \varphi_k(t) \, dt = \int_{\mathbb{R}^n} f(t) \varphi_k(x - t) \, dt
	\end{equation}
	Функции $f_k$ бесконечно дифференцируемые и $f_k \rightarrow f$ равномерно на компактных  подмножествах $\mathbb{R}^ n$.
\end{theorem}
\begin{proof}
	Выпишем разность
	\begin{equation}
		f_k(x) - f(x) = \int_{\mathbb{R}^n} (f(x - t) - f(x)) \varphi_k (t) \, dt
	\end{equation}
	Пусть $f$ равномерно непрерывна в $\delta$ окрестности компакта $K \subset \mathbb{R}^n$ и пусть$|f(x) - f(y)|< \varepsilon$ при $|x - y|< \delta$ в этой окрестности. Выберем $k$ настолько большим, чтобы $1/k < \delta$. Тогда в интеграле $\varphi_k (t)$ отлична от нуля только при $|t|< \delta$, и тогда $|f(x - t) - f(x)|< \varepsilon$, при $x \in K$. Тогда при $x \in K$ верна оценка
	\begin{equation}
		|f_k (x) - f(x)| \leq \varepsilon \int_{\mathbb{R}^n} \varphi_k (x) \, dx  = \varepsilon	
	\end{equation}
	Это показывает равномерную сходимость на компактах. Дифференцируемость можно доказать, используя  дифференцирование интеграла
	\begin{equation}
		\int_{\mathbb{R}^n} f(t) \varphi_k (x - t) \, dt.
	\end{equation}
	по параметру по той же теореме из прошлого сема. Производная при $x \in K$ будет зависеть только от значения $f$ в $1/k$-окрестности $K$, то есть $f$ можно  считать интегрируемой  при дифференцировании по параметру, что позволяет применить теорему.
\end{proof}
\begin{theorem}
	В условиях предыдущей теоремы, если исходная функция $f$ имеет непрерывные производные до $m$-го порядка, то производные $f_k$ до $m$-го порядка равномерно на компактах сходятся к соответствующим производным $f$.
\end{theorem}
\begin{proof}
	Давайте дифференцировать $f * \varphi_k$ по нескольким $x_i$ точно также, как описано выше. Тогда получится
	\begin{equation}
		\frac{\partial^m (f * \varphi_k)}{\partial x_{i_1} \dots x_{i_n}} = \frac{\partial^m f}{\partial x_{i_1} \dots x_{i_n}} *  \varphi_k
	\end{equation}
	Таким образом, последовательность производных свёртки является последовательностью свёрток производной $f$ с теми же функциями $\varphi_k$. А значит для этой последовательности тоже имеет место верна равномерная сходимость к производной $f$.
\end{proof}
\subsection{Лебег!}
\begin{theorem}
	Пусть функция $f : \mathbb{R}^n \rightarrow \mathbb{R}$ имеет конечный интеграл Лебега. Тогда свёртки $f * \varphi_k$ сколь угодно близко приближают $f$ в среднем.
\end{theorem}
\begin{proof}
	Возьмём $\varepsilon > 0$ и представим по теореме из 2ого сема (о приближении ступенчатой в среднем)
	\begin{equation}
		f = g + h
	\end{equation}
	где $g$ --- элементарно ступенчатая и 
	\begin{equation}
		\int_{\mathbb{R}^n} |h(x)| \, dx < \varepsilon
	\end{equation}
	Тогда по теореме \ref{th1} 
	\begin{equation}
		\int_{\mathbb{R}^n} |h * \varphi_k| \, dx < \varepsilon
	\end{equation}
	Что значит, что если будет так, что 
	\begin{equation}
		\int_{\mathbb{R}^n} |g - g * \varphi_k| \, dx < \varepsilon
	\end{equation}
	То будет выполняться 
	\begin{equation}
		\int_{\mathbb{R}^n} |f - f * \varphi_k| \, dx \leq	\int_{\mathbb{R}^n} |h(x)| \, dx + \int_{\mathbb{R}^n} |h * \varphi_k| \, dx + \int_{\mathbb{R}^n} |g - g * \varphi_k| \, dx < 3 \varepsilon
	\end{equation}
	Таким образом, достаточно доказать утверждение для элементарно ступенчатой $g$. Раскладывая $g$ в сумму характеристических функций параллелепипеда с некоторыми коэффициентами, можно видеть, что достаточно доказать утверждение для одной характеристической функции параллелепипеда $\chi_P$. Но разность$\chi_P - \chi_P * \varphi_k$ будет отлична от нуля только в $1/k$-окрестности $\partial P$ и будет там по модулю не более 1, то естьпосле интегрирования модуля разности мы получим не более $\mu(U_{1/k}(\partial P))$. Прямым вычислением можно убедиться, что эта мера стремится к нулю при $k \rightarrow \infty$
\end{proof}
Если говорить проще, то мы смотрим на одну ступеньку и говорим, что ее характеристическая функция отлично приближается свертками. Причем мера точности приближения будет обратно пропорциональна $k$ $\rightarrow$ всё по кайфу.

\part{Дифференцируемые отображения и криволинейные системы координат}
\section{Дифференцируемые отображения и производная композиции отображений}
\subsection{Дифференцируемые отображения}
\begin{definition}[Дифференцируемое отображение]
	Отображение $f : U \rightarrow \mathbb{R}^m$, где $U \subset \mathbb{R}^n$ и открытое, называется дифференцируемым, если представимо как 
	\begin{equation}
		f(x) = f(x_0) + D f_{x_0}(x - x_0) + o(|x - x_0|)
	\end{equation}
	при $x \rightarrow x_0$\\
	где $D f_{x_0} :  \mathbb{R}^n \rightarrow \mathbb{R}^m$ --- линейное отображение, называемое производной в точке $x_0 \in U$.
\end{definition}[Непрерывно дифференцируемое отображение]
\begin{definition}
	 $f : U \rightarrow \mathbb{R}^m$ называется Непрерывно дифференцируемым,  если $\forall x_0 \in U \, \exists D f_{x_0}$, которое непрерывно и непрерывно зависит от $x_0 \in U$.
\end{definition}
Вот эта вот $D$ де-факто --- матрица $m \times n$, в которой каждая ячейка выглядит как $ \left( \frac{\partial f_i}{\partial x_j} \right)$, и для проверки последнего определения достаточно проверить все эти ячейки на непрерывность.
\subsection{Норма матрицы}
Докажем существование "нормы" у матриц линейных отображений:
\begin{lemma}
	$\forall$ линейного $A : \mathbb{R}^n \rightarrow \mathbb{R}^m$ $\exists ||A|| \in \mathbb{R} т.ч. \, \forall x \in \mathbb{R}^n$
	\begin{equation}
		|Ax| \leq ||A|| \cdot |x|
	\end{equation}
\end{lemma}
\begin{proof}
	$Ax$ непрерывно зависит от $x$. Рассмотрим $n-1$-мерную сферу $S^{n-1} = \bigl\{ x \in \mathbb{R}^n \bigm| |x| = 1 \bigl\}$ \\
	$S^{n-1}$ компактно $\rightarrow$ $|Ax|$ достигает максимума на $S^{n-1}$. \\
	Пусть $\max\limits_{|x| = 1} |Ax| = ||A||\in \mathbb{R}$.\\
	$|Ax| \leq ||A|| \cdot |x|$ верно при $|x| = 1$	. При $x = 0$ всё так же очевидно, при $y = tx$ всё будет очевидно после вынесение $t$ за скобки везде.
\end{proof}
\subsection{Производная композиции}
\begin{theorem}
	Пусть у нас есть $f : U \rightarrow \mathbb{R}^m$ и $g : V \rightarrow \mathbb{R}^k$, где $U \in \mathbb{R}^n$, а $V \in \mathbb{R}^m$. Обозначим также $f(x_0) = y_0 \in V, \, x_0 \in U$\\
	Пусть также $f$ дифференцируема в $x_0$ и $g$ дифференцируема в $y_0$. Тогда $g \circ f$ дифференцируемо в $x_0$ и $D(g \circ f)_{x_0} = D g_{y_0} \circ D f_{x_0}$
\end{theorem}
\begin{proof}
	Обозначим $A = D f_{x_0}$ и $B = D g_{x_0}$. Тогда
	$$f(x) = f(x_0) + A(x-x_0) + o(|x-x_0|)$$
	$$g(x) = g(y_0) + B(y-y_0) + o(|y-y_0|)$$
	\begin{equation}
		g(f(x)) = g(f(x_0)) + B(A(x-x_0) + o(|x-x_0|)) + o(A(|x-x_0|) + o(|x-x_0|)) 
	\end{equation}
	это же выражение равно
	\begin{equation}
		g(f(x)) = g \circ f (x_0) + B \cdot A (x - x_0) + B o(|x-x_0|) + o(A|x-x_0|) 
	\end{equation}
	которое используя тот факт, что $C o (x) = o(C x) = o (x)$ преобразовывается как:
	\begin{equation}
		g(f(x)) = g \circ f (x_0) + B \cdot A (x - x_0) + o(|x-x_0|)
	\end{equation}
\end{proof}
\section{Теорема о существовании обратного отображения. Локальные системы криволинейных координат.}
Сразу скажу, что здесь много и долго, настройтесь на это. Ну а теперь начнем с небольшой, простенькой =) леммы.
\begin{lemma}
	Пусть открытое множество $U \subset \mathbb{R}^n$ выпукло. Для непрерывно дифференцируемого отображения $\varphi : U \rightarrow \mathbb{R}^m$ найдётся непрерывное отображение $A : U \times U \rightarrow \mathcal{L} (\mathbb{R}^n,\mathbb{R}^m)$, такое что для любых $x',x'' \in U$
	\begin{equation}
		\varphi(x') - \varphi(x'') = A(x', x'')(x'' - x')
	\end{equation}
	и $A(x, x) = D \varphi_x$. Здесь $ \mathcal{L} (\mathbb{R}^n,\mathbb{R}^m)$ --- линейные отображения из $\mathbb{R}^n$ в $\mathbb{R}^m$ с топологией в пространства матриц $\mathbb{R}^{nm}$.
\end{lemma}
\begin{proof}
	Рассмотрим такую $f(t) = \varphi (t x'' + (1 - t) x')$ Тогда очевидно, что 
	\begin{equation}
		\varphi(x'') - \varphi(x') = \int_0^1 \frac{\partial f}{\partial t} \, dt = \int_0^1 D \varphi_{t x'' + (1 - t) x'}  (x'' - x') \, dt
	\end{equation}
	(Взяли производную сложной функции)\\
	Теперь мы скажем, что наша матрица $A(x', x'')$ это именно этот интеграл
	\begin{equation}
		A(x', x'') = \int_0^1 D \varphi_{t x'' + (1 - t) x'}  (x'' - x') \, dt
	\end{equation}
	(так как $(x'' - x')$ не зависит от $t$, то имеем право вынести за интеграл)\\
	Непрерывность $A$ следует из равномерной непрерывности подынтегрального выражения по переменным $x'$ и $x''$, рассматриваемым как параметры, меняющиеся в рамкахнекоторого компакта $K \subset U \times U$, содержащего маленькую окрестность данной пары$(x', x'')$. При изменении $x'$ и $x''$ не более чем на $\delta  > 0$ из равномерной непрерывности
	значение под интегралом будет меняться не более чем на $\varepsilon > 0$, а значит и сам интеграл будет меняться не более чем на $\varepsilon$. При $x'=x''=x$ из явной формулы мы будем иметь $A(x,x) =D \varphi_x$. \\
	Вроде как это утверждение следует из теорем прошлого сема, так что не нужно бояться, что попросят доказать (хотя тут всего-то равномерная непрерывность интеграла).
\end{proof}
И собственно теперь поговорим о том, ради чего собрались:
\begin{theorem}
	Если отображение $\varphi : U \rightarrow \mathbb{R}^n$ непрерывно дифференцируемо в окрестности точки $x$ и его дифференциал $D \varphi_x$ является невырожденным линейным преобразованием, то это отображение взаимно однозначно отображает некоторую окрестность$V \ni x$ на окрестность $W \ni y$, где $y=\varphi(x)$, и обратное отображение $\varphi^{-1}  :  W \rightarrow V$ тоже непрерывно дифференцируемо.
\end{theorem}
\begin{proof}
	После сдвига координат будем считать, что мы работаем в окрестности точки $x= 0$ и $y=\varphi (x) = 0$. Заменив $\varphi$ на его композицию с линейным отображением $D\varphi^{-1}_0$, будем считать, что $D\varphi$ в нуле является единичным линейным преобразованием, запишем тогда
	\begin{equation}
		\varphi (x) = x + \alpha (x)
	\end{equation}
	Что тут произошло? Мы хотим работать так, чтобы было удобно, поэтому делаем сдвиг (лин замена, ничего не портит) и приводим матрицу преобразования в нуле к единичной численно (по теореме о производной композиции), всё хорошо, потому что там тоже лин. преобразование. Как-то так.\\
	Тогда $D\alpha = D\varphi - id$  в нуле --- нулевой оператор, а в его окрестности очень мал, мал настолько, что верна такая оценка
	\begin{equation}
		|D \alpha (\upsilon) \leq ||D \alpha || \cdot |\upsilon| \leq 1/2 |\upsilon|.
	\end{equation}
	Тогда мы можем применить ту самую лемму в $\delta$-окрестности нуля:
	\begin{equation}
		|\alpha(x'') - \alpha(x') | = |\int_0^1 \frac{d}{dt} \alpha ((1 - t) x' + t x'') \, dt \, \,| =  |\int_0^1 D \alpha_{(1 - t) x' + t x''}  (x'' - x') \, dt \, \,| \leq 1/2 |x'' - x'|
	\end{equation}
	Далее начнем решать задачу
	$$ x = y - \alpha(x) = f(x, y)$$
	При $|y| \leq \delta/2$ и $|x| \leq \delta$ из-за предыдущего неравенства на $\alpha$ получим
	$$|f(x, y) | \leq \delta $$
	И, что самое важное, наше отображение сжимаемое, то есть
	\begin{equation}
		|f(x'', y) - f(x', y)| \leq 1/2 |x'' - x'|
	\end{equation}
	Далее по индукции попробуем решить уравнение, положим $\psi_1 (y) = 0$, далее определим
	$$\psi_k (y) = f(\psi_{k-1}(y), y)$$
	В силу того, что отображение сжимаемое выполняется
	\begin{equation}
		|\psi_{k+1}(y) - \psi_k(y)| = |f(\psi_k(y), y) - f(\psi_{k-1}(y), y)| \leq 1/2 | \psi_k(y) - \psi_{k-1}(y)|
	\end{equation}
	Откуда по индукции можно понять, что 
	\begin{equation}
		|\psi_{k+1}(y) - \psi_k(y)| \leq \delta 2^{2-k}
	\end{equation}
	То есть $\psi_k(y)$ сходятся к некоторому непрерывному отображению $\psi(y)$ непрерывно по признаку Вейерштрасса и переходя к пределу $k \rightarrow \infty$ в определении $\psi_k$ получим
	\begin{equation}
		\psi(y) = f(\psi(y), y) = y - \alpha(\psi(y)) = y - \varphi(\psi(y)) + \psi(y)
	\end{equation}
	То есть, $y = \varphi(\psi(y))$. Из того, что $\alpha$ сжимаемое также следует, что $\forall y: |y|\leq \delta/2 $ найдётся не более одного $x: |x|\leq \delta$, для которого $\varphi(x) =y$, и на самом деле мы его уже нашли как $x=\psi(y)$.Взяв окрестность $W \ni y$, меньшую по сравнению с $\delta/2$, и взяв открытое $V=\varphi^{-1}(W)$ мы видим, что $\varphi$и$\psi$являются взаимно обратными на этих окрестностях. Установим дифференцируемость $\psi$. По "простенькой" лемме можно написать.
	\begin{equation}
		\varphi(x) = \varphi(x) - \varphi(0) = A(x)x
	\end{equation}
	где линейный оператор $A(x)$ непрерывно зависит от $x$и равен $id$  при $x= 0$. Подставим в эту формулу $x=\psi(y)$ и получим
	\begin{equation}
		y = A(\psi(y))\psi(y)  \Rightarrow \psi(y) = A(\psi(y))^{-1}y
	\end{equation}
	Где линейный оператор $B(y)  =A(\psi(y))^{-1} $ непрерывен по $y$ и равен тождественному при $y= 0$. Из выражения $\psi(y) =B(y)y$ тогда следует дифференцируемость $\psi$ в нуле с дифференциалом $B(0)$, дифференцируемость в остальных точках проверяется послесдвига начала координат в соответствующую точку повторением тех же рассуждений.
\end{proof}
Добавить можно лишь, что тут важна невырожденность матрицы Якоби ($D\varphi$).
\begin{definition}
	Криволинейной системой координатв окрестности точки $p \in \mathbb{R}^n$ мы будем называть набор таких функций, которые являются координатами гладкого отображения окрестности $p$ на некоторое открытое множество в $\mathbb{R}^n$с гладким обратным отображением.
\end{definition}
По теореме об обратном отображении для того, чтобы гладкие $y_1, \dots, y_n$ в некоторой окрестности $p$ давали КСК необходима невырожденность матрицы Якоби в точке $p$, иначе говоря, линейная независимость дифференциалов $dy_1, \dots, dy_n$ в точке $p$.
\section{Теоремы о системе неявных функций, определяемых системой уравнений (случай гладких уравнений).}
\begin{theorem}
	Пусть функции $f_1, \dots, f_k$ непрерывно дифференцируемы в окрестности $p \in \mathbb{R}^n$ и определитель
	$$det \left( \frac{\partial f_i}{\partial x_j} \right)^k_{i, j =1}$$
	не равен нулю в этой окрестности. Пусть также $f_i(p) = y_i$. Тогда найдётся окрестностьточки $p$ вида $U \times V,U \subset \mathbb{R}^k,V \subset R^{n-k}$, такая что в этой окрестности множество решений системы уравнений
	$$f_1(x) = y_1, \dots, f_k(x) = y_k$$
	совпадает с графиком непрерывно дифференцируемого отображения $\varphi : V \rightarrow U$, заданного в координатах как
	\begin{align*}
		x_1 &= \varphi_1(y_1, \dots, y_k, x_{k+1}, \dots, x_n) \\
		\dots \\
		x_k &= \varphi_1(y_1, \dots, y_k, x_{k+1}, \dots, x_n) 
	\end{align*}
\end{theorem}
Пояснение: тут и далее $x_i$ --- функции, которые возвращают от точки $p$ одну координату, как бы это очевидно не было. и первые k аргументов - константы, потому являются параметрами.
\begin{proof}
	Условия теоремы означают, что дифференциалы $$df_1, \dots, df_k, dx_{k+1}, \dots, dx_n$$ являются линейно независимыми и из функций $f_1, \dots, f_k, x_{k+1}, \dots, dx_n$ можно составить отображение, локально имеющее непрерывно дифференцируемое обратное, то есть они дают криволинейную систему координат в окрестности $p$. Следовательно, в этой окрестности старые координаты $x_1,\dots,x_k$ можно непрерывно дифференцируемо выразить через новые координаты
	$$x_i = \varphi_i (f_1, \dots, f_k, x_{k+1}, \dots, x_n)$$
	и поставить в этом выражении вместо $f_i$ константы $y_i$.\\
	Это рассуждение доказывает, что множество решений системы уравнений содержится в графике отображения $\varphi : V \rightarrow U$ при достаточно малых $V$ и $U$, таких что $\varphi(V) \in U$. Но и обратное верно, так как значения $f_1, \dots, f_k$ на точке вида
	$$(\varphi_1(y_1,\dots,y_k,x_{k+1},\dots,x_n),\dots,\varphi_k(y_1,\dots,y_k,x_{k+1},\dots,x_n),x_{k+1},\dots,x_n)$$
	обязаны совпадать с $y_1,\dots,y_k$, так как $\varphi_i$ были выбраны как компоненты отображения,обратного к отображению, описанному выше.
\end{proof}
\section{Теорема о расщеплении гладкого отображения на простые гладкие отображения.}
\begin{theorem}
	Если отображение $\varphi$  непрерывно дифференцируемо в окрестности точки $p\in \mathbb{R}^n$ и имеет обратимый $D\varphi_x$, то его можно представить в виде композиции перестановки координат, отражений координат и элементарных отображений, непрерывно дифференцируемо и возрастающим образом меняющих только одну координату $y_i=\psi_i(x_1,\dots,x_n)$.
\end{theorem}
\begin{proof}
	Доказательство этой теоремы имитирует приведение матрицы к гауссовому виду, то есть разложение матрицы в произведение матрицы перестановки, матриц умножений координаты на число, и элементарных матриц. Пусть компоненты $\varphi$ являются функциями $y_1,\dots,y_n$ в окрестности точки $p$. Некоторая $y_i$ имеет ненулевую производную$\frac{\partial y_i}{\partial x_1}$. Переставив $y$ (и запомнив эту перестановку) мы можем считать, что это $y_1$. Поменяв при необходимости знак $y_1$, можно считатьэту производную положительной. Тогда $y_1,x_2,\dots,x_n$ (в силу нетривиальности якобиана) дают криволинейную систему координат в некоторой окрестности $p$ и эта система отличается от исходной возрастающей заменой первой координаты. Далее какая-то из оставшихся $y_2,\dots,y_n$ уже в новой системе координат $y_1,x_2,\dots,x_n$ имеет ненулевую $\frac{\partial y_i}{\partial x_2}$, иначе $dy_i(i=  1,\dots,n)$ не были бы линейно независимыми. Переставив $y$ (и запомнив и эту перестановку), можно считать, что это $y_2$. Также можно считать эту производную положительной, поменяв при необходимости знак $y_2$. Тогда можно заменить$y_1,x_2,\dots,x_n$ на систему координат $y_1,y_2,\dots,x_n$. Делая в том же духе $n$ раз, мы сделаем $n$ замен координат (отображений), меняющих возрастающим образом только одну координату, а в конце нам останется поменять знаки у некоторых $y_i$ и переставить их.
\end{proof}
\part{Дифференциал, гессиан и исследование функции на экстремум}
\section{Дифференциал функции как линейный функционал. Корректность определения второго дифференциала (гессиана) функции как квадратичной формы на касательных векторах для случая, когда первый дифференциал функции равен нулю.}
\subsection{Дифференциал функции как линейный функционал}
\begin{definition}
	Пусть $U \subset \mathbb{R}^n$ --- открытое множество. Отображение $f : U \rightarrow \mathbb{R}^m$ называется дифференцируемым в точке $x_0 \in U$, если
	\begin{equation}
		f(x) = f(x_0) + D f_{x_0} (x - x_0) + o(|x - x_0|), \, x \rightarrow x_0
	\end{equation}
	где $D f_{x_0} :  \mathbb{R}^n \rightarrow \mathbb{R}^m$ является линейным отображением. Далее оговариваемся, что для функций $f : U \rightarrow \mathbb{R}^m$  мы вводим обозначение $Df_x = df_x$ и называть это дифференциалом функции. По определению это линейная форма из $\mathbb{R}^n \rightarrow \mathbb{R}$.
\end{definition}
\subsection{Корректность определения второго дифференциала (гессиана) функции как квадратичной формы на касательных векторах для случая, когда первый дифференциал функции равен нулю}
Тут скорее всего идет речь о том, что при замене координат гессиан ведет себя как квадратичная форма. 
\begin{lemma}
	Если $df_{x_0} = 0$, то при любой замене координат $x = \varphi(t)$ гессиан в точке $x_0 = \varphi(t_0)$ меняется так:
	\begin{equation}
		d_2(f \circ \varphi)_{t_0}(\Delta t) = d_2 f_{x_0} (D \varphi_{t_0}(\Delta t))
	\end{equation}
\end{lemma}
\begin{proof}
	Для нахождения элементов второго дифференциала (как матрицы) надо дифференцировать композицию один раз, а потом ещё один раз. Помимо выписанных слагаемых со вторыми производными $f$ и первыми производными $\varphi$ могли бы появиться слагаемые с первыми производными $f$ и вторыми производными $\varphi$. Но по условию в точке $x_0$ первые производные $f$ равны нулю, а значит выражение содержит только вторые производные $f$.
\end{proof}
\section{Локальные максимумы и минимумы функций многих переменных. Необходимое условие экстремума непрерывно дифференцируемой функции.}
\subsection{Локальные максимумы и минимумы функций многих переменных.}
\begin{definition}[Локальный экстремум функции]
	Точка $p$ называется локальным экстремумом функции $f$, если является строгим ее экстремумом (max || min) при ограничении $f$ на некоторую окрестность $p$.
\end{definition}
\subsection{Необходимое условие экстремума непрерывно дифференцируемой функции.}
\begin{theorem}
	Если $f$ дифференцируема в точке $p$ и имеет локальный экстремум в $p$, то $df_p = 0$.
\end{theorem}
\begin{proof}
	$f(x_0 + t e_i) = g(t)$ имеет экстремум в $t = 0$, откуда получаем
	\begin{equation}
		\frac{\partial g}{\partial t} = \frac{\partial f}{\partial x_i}(x_0) = 0
	\end{equation}
	$\forall x_i \Rightarrow df_p = 0$ ЧТД.
\end{proof}
\section{Необходимые и достаточные условия экстремума дважды непрерывно дифференцируемых функций.}
\subsection{Необходимое условие экстремума дважды непрерывно дифференцируемой функции}
\begin{theorem}
	Если $f$ дважды непрерывно дифференцируема в окрестности точки $p$, то $d_2 f_p \geq 0$ для минимума и $d_2 f_p \leq 0$ для максимума. 
\end{theorem}
\begin{proof}
	Б.о.о. рассмотрим минимум. Запишем формулу тейлора для $\xi = p + t \upsilon$
	\begin{equation}
		f(p+t\upsilon) = f(p) + \frac{1}{2} d_2 f_p(t \upsilon) + o(t^2 |\upsilon|^2) = f(p) + t^2(\frac{1}{2} d_2 f_p(\upsilon) + o(1))
	\end{equation}
	А так как в точке $p$ экстремум, то при пределе $t \rightarrow 0$ второй дифференциал должен быть положителен или равен нулю.
\end{proof}
\subsection{Достаточное условие экстремума дважды непрерывно дифференцируемой функции}
Сначала докажем одну лемму
\begin{lemma}
	Если квадратичная форма $Q$ положительно определена, то найдётся $\varepsilon > 0$, такой что
	\begin{equation}
		Q(\upsilon) \geq \varepsilon |\upsilon|^2
	\end{equation}
	для любого $\upsilon$
\end{lemma}
\begin{proof}
	Положим $\varepsilon$ равным минимуму $Q$ на единичной сфере. Так как единичная сфера является компактом, а квадратичная форма является непрерывной функцией, то этот минимум достигается и положителен. Тогда неравенство верно для единичной сферы. Из этого следует неравенство для всех векторов, так как при умноже-нии вектора на $t$ обе части неравенства умножаются на $t^2$
\end{proof}
\begin{theorem}
	Если $f$ дважды непрерывно дифференцируема в окрестностиp, $d f_p = 0$ и $d_2 f_p > 0$, то $p$ ---  точка строгого локального минимума. Если неравенство в другую сторону, $d_2 f_p < 0$,  то $p$ --- точка строгого локального максимума.
\end{theorem}
\begin{proof}
	Докажем без ограничения общности для минимума. Для $\xi = p + \upsilon$ запишем по формуле Тейлора с использованием предыдущей леммы:
	\begin{equation}
		f(p+\upsilon) = f(p) + \frac{1}{2} d_2 f_p(\upsilon) + o(|\upsilon|^2) \geq f(p) + \left( \frac{\varepsilon}{2} + o(1) \right) |\upsilon|^2
	\end{equation}
	По определению $o(1)$ при достаточно малом $|\upsilon|$ (независимо от направления $\upsilon$) выражение в скобках будет положительным.
\end{proof}
\section{Условные экстремумы. Необходимое условие условного экстремума в терминах первых производных. Метод множителей Лагранжа.}
\subsection{Условные экстремумы}
\begin{definition}[Условный экстремум]
	экстремума ограничения функцииfна множество $S$, задаваемое системой непрерывно дифференцируемых (как минимум) уравнений
	$$\varphi_1(x) = \dots = \varphi_m(x) = 0$$
	Также требуется линейная независимость дифференциалов
	$$dim \left< d \varphi_1, \dots, d \varphi_m \right> = m$$
\end{definition}
\subsection{Необходимое условие условного экстремума в терминах первых производных}
\begin{theorem}
	Если $f$ и $\varphi_1, \dots, \varphi_m$ непрерывно дифференцируемы в окрестности $p$, дифференциалы $\varphi$ линейно не зависимы и $f$ имеет условный экстремум в $p$ при условии $\varphi_1(x) = \dots = \varphi_m(x) = 0$, то в точке $p$ выполняется
	\begin{equation}
		d f_p = \lambda_1 d \varphi_{1, p} + \dots + \lambda_m d \varphi_{m,p} 
	\end{equation}
	для некоторых $\lambda_1, \dots, \lambda_m$.
\end{theorem}
\begin{proof}
	Заметим, что нахождение $df$ в линейной оболочке $d\varphi_i$ инвариантно относительно криволинейных замен координат по формуле дифференциала композиции. Согласно этой формуле, при замене координат строка чисел $\frac{\partial f}{\partial x_i}$ получается из строки чисел $\frac{\partial f}{\partial y_j}$ умножением справа на матрицу с элементами $\frac{\partial y_j}{\partial x_i}$.\\
	Тогда мы можем считать $\varphi_1=y_1,\dots,\varphi_m=y_m$ в системе координат $y_1,\dots,y_n$. В этом случае мы имеем экстремум функции по остальным переменным при условии $y_1= \dots =y_m= 0$, что даёт равенства$\frac{\partial f}{\partial y_i}= 0$ при $i > m$. Тогда можно составить линейную комбинацию
	\begin{equation}
		d f_p = \lambda_1 d y_1 + \dots + \lambda_m d y_m
	\end{equation}
\end{proof}
\subsection{Метод множителей Лагранжа}
\begin{enumerate}
	\item Составим функцию Лагранжа в виде линейной комбинации функции $f $ и функций $\varphi_i$, взятых с коэффициентами, называемыми множителями Лагранжа --- $\lambda_i$:
	\begin{equation}
		L(x, \lambda) = f(x) + \sum_{i = 1}^m \lambda_i \varphi_i(x)
	\end{equation}
	\item Составим систему из $n + m$ уравнений, приравняв к нулю частные производные функции Лагранжа $L(x, \lambda)$ по $x_j$ и $\lambda_i$.
	\item Если полученная система имеет решение относительно параметров $x'_j$ и $\lambda'_i$, тогда точка $x'$ может быть условным экстремумом, то есть решением исходной задачи. Заметим, что это условие носит необходимый, но не достаточный характер.
\end{enumerate}
\section{Необходимые и достаточные условия условного экстремума с использованием вторых производных.}
\subsection{Необходимое условие условного экстремума с использованием вторых производных}
\begin{theorem}
	Если $f$ и $\varphi_1,\dots,\varphi_m$ дважды непрерывно дифференцируемы в окрестности $p$, выполняется линейная независимость дифференциалов и $f$ имеет условный экстремум в $p$ при условии $\varphi_1(x) = \dots = \varphi_m(x) = 0$, то
	$$d f_p = \lambda_1 d \varphi_{1, p} + \dots + \lambda_m d \varphi_{m, p}$$
	и второй дифференциал функции Лагранжа положительно полуопределён (для минимума)или отрицательно полуопределён (для максимума) на векторах $\upsilon$, удовлетворяющих линейным уравнениям
	$$d \varphi_{1, p}(\upsilon) = \dots = d \varphi_{m, p}(\upsilon) = 0$$
\end{theorem}
\begin{proof}
	Заметим, что при выполнении условий $\varphi_1(x) = \dots = \varphi_m(x) = 0$ функции $f$ и $L$ равны. Но функция Лагранжа удобнее тем, что $dL_p= 0$. Лемма о корр. кв формыпозволяет в этом случае считать $d_2L_p$ корректно определённой квадратичной формой и сделать замену координаттак, чтобы $\varphi_1=y_1,\dots, \varphi_m=y_m$, при этом $d_2L_p$ преобразуется так, как положено преобразовываться квадратичной форме при линейном преобразовании, которое является производной замены координат.\\
	После замены координат мы фактически рассматриваем функцию $L$ при фиксированных первых $m$ переменных. Допустимые приращения соответствуют векторам $\upsilon$, первые $m$ координат которых равны нулю, то есть
	\begin{equation}
		d y_1(\upsilon) = \dots = d y_m(\upsilon) = 0
	\end{equation}
	Но тогда теорема о необходимом условии экстремума без условий показывает, что $d_2L_p$должна быть полуопределена на допустимых приращениях $\upsilon$, что после обратной замены координат превращается в утверждение теоремы.
\end{proof}
\subsection{Достаточное условие условного экстремума с использованием вторых производных}
\begin{theorem}
	Если $f$ и $\varphi_1,\dots,\varphi_m$ дважды непрерывно дифференцируемы в окрестности $p$, выполняется линейная независимость дифференциалов
	$$d f_p = \lambda_1 d \varphi_{1, p} + \dots + \lambda_m d \varphi_{m, p}  \Leftrightarrow d L_p = 0$$
	и второй дифференциал функции Лагранжа положителен (для минимума) или отрицателен (для максимума) на ненулевых векторах $\upsilon$, удовлетворяющих линейным уравнениям
	$$d \varphi_{1, p}(\upsilon) = \dots = d \varphi_{m, p} (\upsilon) = 0$$
	то $f$ имеет строгий условный экстремум в $p$ на ограничении $S$.
\end{theorem}
\begin{proof}
	Аналогично предыдущей теореме, взяв $\varphi_i$ за первые $m$ координат ивоспользовавшись леммой о корр. опр. гессиана, мы сводим задачу к случаю экстремума без условий.
\end{proof}
\part{Векторы и дифференциальные формы первой степени}
\section{Касательные векторы к открытому подмножеству $\mathbb{R}^n$ в точке. Определение через дифференцирование функций в точке и явный вид}
\subsection{Касательные векторы к открытому подмножеству $\mathbb{R}^n$ в точке}
Докажем лемму, которая будет нужна далее:
\begin{lemma}
	Всякую гладкую функцию, определённую в некоторой окрестности $x_0 \in \mathbb{R}^n$, в возможно меньшей окрестности $x_0$ можно представить в виде
	\begin{equation}
		f(x) = f(x_0) + \sum_{k = 1}^n (x_k - x_{0,k})g_k(x)
	\end{equation}
\end{lemma}
\begin{proof}
	Это следует из леммы 10 и её доказательства. Бесконечная дифференцируемость функций $g_k$ (которые в той лемме были компонентами отображения $A$) следует из возможности бесконечно дифференцировать определяющий их интеграл по параметрам
\end{proof}
\begin{definition}[Касательные векторы к открытому подмножеству $\mathbb{R}^n$ в точке]
	Определим касательный векторв точке $p \in U$ открытого множества $U\subset \mathbb{R}^n$ как $\mathbb{R}$-линейное отображение $X : C^{\infty}(U) \rightarrow \mathbb{R}$, удовлетворяющее
	\begin{equation}
		X(fg) = X(f)g(p) + X(g)f(p)
	\end{equation}
\end{definition}
\subsection{Определение через дифференцирование функций в точке и явный вид}
Запишем используя Лемму 22, запишем
\begin{equation}
	X(f) = X(f(p)) + \sum_{i = 1}^n (x_i - p_i) X(g_i) + \sum_{i = 1}^n X(x_i - p_i)g_i(p)
\end{equation}
Что равно
\begin{equation}
	X(f) = \sum_{i = 1}^n  X(x_i)g_i(p) = \sum_{i = 1}^n X_i \frac{\partial f}{\partial x_i}
\end{equation}
\section{Касательное пространство в точке и дифференциал отображения как отображение касательных пространств. Векторные поля на открытых областях в $\mathbb{R}^n$.}
\subsection{Касательное пространство в точке}
\begin{definition}
	Касательное пространство к $U$ в точке $p$ состоит из всех касательных векторов в точке $p$ и обозначается как $T_p U$
\end{definition}
\subsection{Дифференциал отображения как отображение касательных пространств}
Можно также корректно определить образ касательного вектора при произвольном гладком отображении $\varphi : U \rightarrow V$, не обязательно обратимом, следующим образом.Пусть у нас есть вектор $X \in T_p U$, $q=\varphi(p)$, тогда прямой образ вектора,$\varphi_*(X)$, определяется по формуле
\begin{equation}
	\varphi_*(X)f = X (f \circ \varphi)
\end{equation}
Можно выписать координаты этого вектора
\begin{equation}
	\varphi_*(X) = \sum_{i=1}^n \sum_{j=1}^m \frac{\partial \varphi_j}{\partial x_i} X_i \frac{\partial}{\partial y_j}
\end{equation}
Таким образом мы можем бескоординатно определить производную отображения $\varphi$ в точке $p$ как линейное отображение $\varphi_* : T_pU \rightarrow T_qV$ при $q=\varphi(p)$. Его также можно обозначать как $D\varphi_p$, так как в силу своего выражения в координатах оно на самом деле совпадает с введённый ранее производной в смысле линейного приближения отображения, что проясняет геометрический смысл конструкции прямого образа вектора.
\subsection{Векторные поля на открытых областях в $\mathbb{R}^n$}
\begin{definition}[Векторное поле на открытом $U \subset \mathbb{R}^n$]
	$X$ --- гладкое сопоставление $\forall p \in U$ вектора в точке $p$. Вектор в $p \in U$ --- $T_p U$.
\end{definition}
\begin{definition}[Векторное поле на открытом $U \subset \mathbb{R}^n$]
	Векторное поле на $U^{откр} \subset \mathbb{R}^n$ --- это $\mathbb{R}$-линейное $X: C^{\infty} (U) \rightarrow C^{\infty} (U)$ т.ч. 
	\begin{equation}
		X(f \cdot g) = X(f) g + X(g) f
	\end{equation}
\end{definition}
\section{Дифференциальные формы первой степени и дифференциалы функций. Замена координат в дифференциальной форме первой степени.}
\subsection{Дифференциальные формы первой степени и дифференциалы функций}
Для всякой функции $f \in C^{\infty}(U)$ её дифференциал как отображения $U \rightarrow \mathbb{R}$ можно считать формой первой степени в соответствии с формулой 
\begin{equation}
	df(X) = X(f)
\end{equation}
действительно, это выражение линейно относительно умножения $X$ на бесконечно гладкие функции и в координатах компоненты $df$ оказываются равны $\frac{\partial f}{\partial x_i}$, то есть это уже известный нам дифференциал функции, но определённый по-новому.
\subsection{Замена координат в дифференциальной форме первой степени}
Дифференциалы координатных функций $dx_1,\dots,dx_n$ в любой точке дают базис пространства $T^*_p U$, двойственный к базису $\frac{\partial}{\partial x_1}, \dots, \frac{\partial}{\partial x_n}$ в смысле
\begin{equation}
	dx_i \left( \frac{\partial}{\partial x_j} \right) = \frac{\partial x_i}{\partial x_j} = \delta_{ij}
\end{equation}
По этому базису можно разложить любую форму в точке, а применяя это во всех точках области $U\subset \mathbb{R}^n$ обнаруживаем, что всякая дифференциальная форма на $U$ первой степени выражается как
\begin{equation}
	\alpha_1 d x_1 + \dots + \alpha_n dx_n
\end{equation}
где $\alpha_i \in C^{\infty}(U)$. При замене координат компоненты дифференциальной формы первой степени ведут себя так же, как компоненты дифференциала функции, то есть преобразование от новой к старой системе координат выглядит как
\begin{equation}
	\alpha_j = \sum_j \widetilde{\alpha_i} \frac{\partial y_i}{\partial x_j}
\end{equation}
\part{Дифференциальные формы высших степеней}
\section{Дифференциальные формы произвольной степени на открытых множествах в $\mathbb{R}^n$, их определение и свойства.}
\subsection{Определение}
Сначала введем некоторые обозначения:\\
$V$ --- векторное пространство.\\
$V^*$ --- двойственное $V = \bigl\{ \text{линейное } \lambda : V \rightarrow \mathbb{R} \bigl\}$\\
существует такая полилинейная форма $\omega$, что 
$$\omega : (\upsilon_1, \dots, \upsilon_k)  \mapsto \omega  (\upsilon_1, \dots, \upsilon_k) \in \mathbb{R}$$
$$\Lambda^kV^* \subset \omega, k\text{-ой степени}$$
с правилом кососимметрии
\begin{definition}
	В $U \subset \mathbb{R}^n$ $\Omega^k(U)$ (дифф. форма $k$ой степени) --- это выбор формы в $\Lambda^k T_p^* U$ $\forall p \in U$ гладко зависящего от $p$
\end{definition}
\begin{definition}
	$\omega \in \Omega^k(U)$ --- это $C^{\infty}(U)$-линейное сопоставление набору из $k$ векторных полей $f \in C^{\infty}(U)$ + кососимметричное.
\end{definition}
$$\omega (f_1 X_1, \dots, f_k X_k) = f_1,\dots, f_k\omega(X_1, \dots, X_k)$$
$$\omega(X_{\sigma(1)}, \dots, X_{\sigma(k)}) = sgn(\sigma) \cdot \omega (X_1, \dots, X_k)$$
\subsection{Свойства}
\begin{lemma}
	Значение выражения $\alpha(X_1,\dots,X_k)$ в точке $p$ зависит только от значений векторных полей $X_i$ в точке $p$.
\end{lemma}
\begin{proof}
	Представим каждое поле $X_i$ в виде $\Sigma_j X^j_i \frac{\partial}{\partial x_j}$. Применив линейность $\alpha$, разложим выражение на слагаемые, в каждом из которых $\alpha$ применяется к векторным полям $\frac{\partial}{\partial x_i}$ (это будет не зависящая от векторных полей часть), а множители $X^j_i$ вынесены из $\alpha$ как функции. Значение этого выражения в точке $p$ будет зависеть только отзначений $X^j_i$ в точке $p$.
\end{proof}
вроде больше свойств нет $\dots$
\end{document} % конец документа
