\section{Основы фотометрии. Яркость источника, освещённость изображения. Теорема о сохранении яркости оптической системой.}
\subsection{Основы фотометрии}
\subsubsection{Поток энергии}
Обозначим $\Phi_Э$ поток энергии (энергию в единицу времени)
$$\Phi_Э = \int\limits_{0}^{\infty} \varphi(\lambda) d \lambda$$
где $\varphi(\lambda)$ --- спектральная плотность энергии (рассчитана на единичный интервал длины волн)
\subsubsection{Видность}
Шаманская функция $V(\lambda)$ с гауссовским графиком вокруг $\lambda = 555нм$
\subsubsection{Световой поток}
Для характеристики интенсивности света с учётом его способности вызывать зрительные ощущения вводится величина $\Phi$ --- световой поток. Для интервала $d \lambda$ имеем
$$d\Phi = V(\lambda) d \Phi_Э$$
Полный световой поток может быть представлен в виде
$$\Phi = \int\limits_{0}^{\infty}$$
\subsubsection{Сила света}
Световой поток, излучаемый в единичный телесный угол, называется силой света:
$$J = d \Phi / d \Omega$$
Полный световой поток --- $\Phi = 4\pi J$.
\subsubsection{Интенсивность света}
$$I = d \Phi / d S_{\perp}$$
\subsection{Яркость источника}
Это понятие характеризует поверхности и неприменимо к точечным источникам
$$B = \frac{d \Phi}{d S_{\perp} d \Omega} = \frac{d J }{d S_{\perp}} = \frac{d J}{d S \cos \theta }$$
\subsection{Освещённость изображения}
$$E = \frac{d \Phi}{d S}$$
Пусть источник света точечный, тогда величина
$$d \Phi = J d \Omega = J \frac{dS_{\perp}}{r^2}$$
есть поток, падающий на площадку $dS_{\perp}  = dS\cos\theta$, где $\theta$ --- угол падения излучения на поверхность. В результате оказывается, что 
$$E = \frac{d\Phi}{dS} = \frac{J}{r^2} \cos \theta$$
\subsection{Теорема о сохрании яркости оптической системой}
Пусть оптическая система создаёт некоторое изображение светящегося предмета. \\
Найдём сначала яркость изображения, рассматривая последнее, как светящийся объект. Пусть исходный предмет есть квадрат со стороной $y$, а его изображение представляет из себя квадрат со стороной $y'$. Пусть яркость предмета равна $B$. Тогда световой поток, создаваемый предметом, равен
$$d\Phi = B(\theta)d\Omega d S \cos \theta = 2\pi B(\theta) dS\sin\theta\cos\theta d\theta \approx 2\pi B(\theta)dS \theta d \theta$$
Для изображения также
$$d\Phi'= B'(\theta')d\Omega' d S' \cos \theta' = 2\pi B'(\theta') dS'\sin\theta'\cos\theta' d\theta' \approx 2\pi B'(\theta')dS' \theta' d \theta'$$
Считая, что потери энергии малы: $d\Phi = d \Phi'$ и из теоремы Лагранжа-Гельмгольца $ny\theta = n'y'\theta'$.
Тогда 
$$B'(\theta') = \left(\frac{n'}{n}\right)^2 B(\theta)$$
Откуда при равенстве показателей преломления по обе стороны от оптич. системы получаем сохранение яркости Q.E.D.