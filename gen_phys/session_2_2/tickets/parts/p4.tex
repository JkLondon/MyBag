\section{Волновое уравнение. Монохроматические волны. Уравнение Гельмгольца. Комплексная амплитуда. Волновой вектор, фазовая скорость. Плоские и сферические волны.}
\subsection{Волновое уравнение}
Сначала запишем уравнения максвелла и материальные уравнения: 
$$\operatorname{rot} E = -\frac{1}{c} \frac{\partial B}{\partial t}, \qquad \operatorname{div} D = 0$$
$$\operatorname{rot} H = -\frac{1}{c} \frac{\partial D}{\partial t}, \qquad \operatorname{div} B = 0$$
$$D = \varepsilon E \qquad H = \mu B$$
Откуда получаем, что 
$$\nabla^2 E = \frac{\varepsilon \mu}{c} \frac{\partial^2 E}{\partial t^2}$$
что является волновым уравнением.
\subsection{Монохроматические волны}
Монохроматическая волна --- строго гармоническая (синусоидальная) волна с постоянными во времени частотой, амплитудой и начальной фазой. \\
Её уравнение:
$$\nabla^2 \Psi = \frac{\varepsilon \mu}{c} \frac{\partial^2 \Psi}{\partial t^2}$$
Частное решение уравнения:
$$\Psi = u e^{-i \omega t} \qquad \Psi = u e^{i \omega t}$$
где $\omega$ --- частота волны.
\subsection{Уравнение Гельмгольца.}
Если поле меняется по периодическому закону, то волновое переходит в уравнение Гельмгольца:
$$(\Delta+k^2)E = 0$$
\subsection{Комплексная амплитуда}
В трехмере уравнение плоской волны будет иметь вид 
$$E(r,t) = E_1 \exp [i(kr - \omega t + \varphi_0)]$$
где $E_0 = E_1\exp(i \varphi_0)$ будет являться комплексной амплитудой.\\
\subsection{Волновой вектор}
Вектор $k$ называется волновым, $k = 2\pi / \lambda$. Он направлен вдоль распространения волны.
\subsection{Фазовая скорость}
Точки с фиксированной фазой движутся со скоростью
$$\upsilon = \omega / k = c / n$$
\subsection{Плоские и сферические волны}
Плоские и сферические волны являются решениями уравнения максвелла.
Уравнение плоской волны было выше, сферическая --- то же самое, но поделить на $r$.