\subsubsection*{5.13}
Для максимального порядка интерфернционных полос  при $\varphi = 0$ получаем $m_{max} = (2hn - \lambda/2)/\lambda = 1000$. Для минимального порядка при $\varphi = 90^{\circ}$ находим $m_{min} = \left[ 2h(n^2 - 1)^{1/2} - \lambda/2\right] /\lambda = 714$. Допустимую немонохроматичность оцениваем как $\Delta \lambda = \lambda/m_{max} \approx 0,56нм$. Так как зрительная труба установлена на бесконечность, картина наблюдается как бы на бесконечности, поэтому источник может быть любого размера и в любом положении.