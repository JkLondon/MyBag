\subsubsection*{10.21}
\begin{equation}\label{eq10211}
	n^2 = \varepsilon = 1 - 4 \pi N (e^2 / m) / \omega^2 = 1 - \omega_p^2 / \omega^2 = 1 - \nu_p^2 / \nu^2
\end{equation}
\begin{equation}\label{eq10212}
	u  = c^2 / \upsilon = cn
\end{equation}
Из (\ref{eq10211}) и (\ref{eq10212}) получаем
\begin{equation*}
	u = c(1 - \omega_p^2 / \omega^2)^{1/2}
\end{equation*}
для показателя преломления близком к единице находим
\begin{equation*}
	u \approx c \left[1 - Ne^2 / (2\pi m \nu^2)\right]
\end{equation*}
Для разницы времён прихода импульсов получаем
\begin{equation*}
	\Delta t = L(1/u_1 - 1/u_2) \approx LNe^2 / (2\pi mc  \nu_1^2)
\end{equation*}
Отсюда $L = 2\pi cm \nu_1^2 \Delta t / (Ne^2) \approx 6,67 \cdot 10^{20}$ см $\approx 700$ св. лет.