\subsubsection*{T1}

\subsubsection*{a)}
У близорукого человека будет зрение определяться соотношением:

\begin{equation*}
	\frac{1}{L_d} + \frac{1}{b_0} = D_{бл}
\end{equation*}

Требуется, чтобы было:

\begin{equation*}
	\frac{1}{\infty} + \frac{1}{b_0} = D_{ид}. 
\end{equation*}

Таким образом требуемая сила очков будет равна:

\begin{equation*}
	\Delta D = - \frac{1}{L_d} = -2 дптр
\end{equation*}

\subsubsection*{б)}

У дальнозоркого человека оптимальное зрение определится соотношением:

\begin{equation*}
	\frac{1}{L_b} + \frac{1}{b_0} = D_{дал}
\end{equation*}

Требуется, чтобы было:

\begin{equation*}
	\frac{1}{L_0} + \frac{1}{b_0} = D_{необ}. 
\end{equation*}

Таким образом требуемая сила очков будет равна:

\begin{equation*}
	\Delta D = \frac{1}{L_0} - \frac{1}{L_b} = 3 дптр
\end{equation*}