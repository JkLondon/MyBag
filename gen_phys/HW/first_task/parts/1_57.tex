\subsubsection*{1.57}

Яркость изображения в данном случае --- это освещённость получаемого изображения. Обозначая освещённость от Луны на поверхности Земли за $E_0$, для освещённости в глазу (без телескопа) получаем ,$B = E_0 \pi d_з^2 / (4S_0)$, где $S_0$ --- площадь изображения в глазу. Для увеличения в телескопе имеем

\begin{equation*}
	\Gamma = \frac{\alpha}{\beta} = \frac{f_2}{f_1}  = \frac{D}{d} 
\end{equation*}
Отношение площадей изображения в глазу $S/S_0 = (D/d)^2$.\\
Диаметр потока, выходящего из телескопа $d = D/\Gamma$ оказывается меньше диаметра зрачка лишь при $Г = 50$.\\
Пока диаметр больше --- яркость та же (случаи 1 и 2). При $\Gamma = 50$ поток (а следовательно и яркость) уменьшается в 4 раза.