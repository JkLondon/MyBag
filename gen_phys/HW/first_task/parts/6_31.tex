\subsubsection*{6.31}
\begin{equation}\label{eq6311}
	r^2_m \approx 2ah_m \approx abm \lambda / (a + b)
\end{equation}
По условию фокусное расстояние $f = 1/D = 1/(2,5)\ м$. Из {\ref{eq6311}} при $a = \infty$ получаем число зон Френеля $m = r^2 / (\lambda f) = 5,5$. Интенсивность для $5,5$ зон Френеля: $I_2 = 2A_0^2$. Так как фокус линзы от всех элементов фронта, пришедшего  к отверстию, волны приходят в одинаковой фазе, все составляющие вытягиваются в прямую линию. Длина этой линии равна 5,5 полуоборотов окружности радиусом $A_0$, т.е. $A=m\pi A_0$. Интенсивность $I_1 = (m\pi)^2A_0^2$. Отношение интенсивностей $I_1/I_2 = (m\pi)^2 = 150$\\
Отношение размеров пятен можно найти из условия, что в обоих случаях в пятнах сосредоточена одна и та же энергия (проходящая через отверстие в экране) $I_1 d_1^2 = I_2 d_2^2$. Отсюда $d_1/d_2  = \sqrt{150} \approx 12$.