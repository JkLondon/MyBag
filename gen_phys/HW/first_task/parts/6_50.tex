\subsubsection*{6.50}
В сферической сходящейся волне наблюдается такой же эффект, как при использовании собирающей линзы: волны от всех элементов отверстия приходят в одной и той же фазе --- спираль разворачивается в прямую линию. Суммарная амплитуда соответствует длине линии. В данном случае (трёх зон) она равна $I=9 \pi^2 I_0$. Сферическая сходящаяся волна создаёт в плоскости экрана поле практически с одной амплитудой, но со сдвигом по фазе, который можно вычислить по расстоянию от фронта сферической волны до плоскости экрана. Для малых углов схождения это равно приблизительно расстоянию от точки фронта до плоскости экрана по нормали ($x$). Обозначив расстояние от оси экрана $\rho$ и считая радиус волны $\approx R_0$, получим $R_0^2 = (R_0 - x)^2 + \rho^2$, откуда $x = \rho^2/(2R_0$. Отсчитывая фазу волн от фаз волны в центре отверстия, получаем в зависимости от $\rho$ фазы волн $\phi_1 = (2\pi / \lambda)\rho^2/(2R_0)$. Аналогичным образом для точки $z$ на оси экрана находим $\phi_2 =  (2\pi / \lambda)\rho^2/(2z)$. Относительный сдвиг фаз 
\begin{equation*}
	\varphi = \varphi_2 - \varphi_1 = (\pi / \lambda) \rho^2 (1/z - 1/R_0)
\end{equation*}
Граница зон Френеля (при z $z \geq R_0$) определяется из условия $\varphi = -m\pi$. Поэтому 
\begin{equation*}
	\rho^2_m = m \lambda z R_0 / (z - R_0)
\end{equation*}
Используя условие, что при плоской волне для $R$ отверстие открывает три зоны Френеля, получаем: $D^2/4 = m \lambda z R_0 / (z - R_0) = 3 \lambda R_0$, откуда
\begin{equation*}
	m = 3(z - R_0) / z
\end{equation*}
При $z = R_0$ имеем $m = 0$. При $z  = 3R_0$ получаем $m = 2$. Две зоны Френеля дают минимум. Других минимумов нет, так как для чётных $m$ получаются отрицательные значение $z$.