\subsubsection*{7.53}
В соответствии с (\ref{eq7101}) находим площадь дифракционного пятна. Считая, что мощность равномерно распределяется по пятну, находим её часть, которая попадает на зрачок. Она должна быть больше той, которую видит глаз, $N \pi(d^2/4) /[(\pi/4) (L \cdot 2,24 \lambda D)^2] \geq n h \nu$. Откуда 
\begin{equation*}
	L \leq (dD/c)[N\nu /(nh)]^{1/2} \approx 3,2 \cdot 10^8 км
\end{equation*}