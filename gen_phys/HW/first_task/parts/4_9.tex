\subsubsection*{4.9}
Показатель преломления воздуха $n$ растет с увеличением плотности, которая пропорциональна давлению. При малых изменениях можно считать в данном случае для кювет длиной $l$ что $(n-1)l = a \Delta p$, где $a$ --- постоянная величина. Используя  $I = 2I_0\left[ 1 + \cos (k \Delta r)\right] $, получаем
\begin{equation*}
	 I = 2I_0( 1 + \cos  \left[ k (n-1)l\right] )
\end{equation*}
Подставляя $k = \omega / c$ и интегрируя по спектру находим 
\begin{equation*}
	I = 2(I_0/\Delta \omega) \int_{\omega_1}^{\omega_2} \left[ 1 + \cos (\omega a \Delta  p / c) \right] d\omega 	
\end{equation*}
\begin{equation*}
	I = 2I_0\left( 1 + \left[2c/(\Delta \omega a \Delta p)\right]\cos \left[(\omega_1 + \omega_2) a \Delta p /(2c)\right] \sin \left[\Delta \omega a \Delta p / (2c)\right] \right)
\end{equation*}
Обозначив $(\omega_1 + \omega_2)/2 = \omega$ получаем условие первого минимума $\omega a \Delta p_1 /c =\pi$. Картина исчезает, когда аргумент синуса становится равным $\pi = \Delta \omega a \Delta p_2 / (2c)$. Исключая $a$, находим 
\begin{equation*}
	\Delta p_2 = \Delta p_1 2 \omega/ \Delta \omega = 200 мм \ рт. \ ст.
\end{equation*}
