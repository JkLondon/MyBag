\subsubsection*{7.59}
Числовая апертура $a = n \sin u$, где $u$ --- апертура из точки предмета, лежащей на оптической оси. Для повышения  числовой апертуры применяют иммерсию, т. е. жидкость с возможно высоким показателем преломления, заполняющую пространство между покровным стеклом над рассматриваемым предметом и объективом. Из \ref{eq7102} получаем 
\begin{equation*}
	\varphi = \delta / f   = 1,22 \lambda / (nD) = 1,22 \lambda / (2fn \sin u)
\end{equation*} 
Отсюда
\begin{equation*}
	\delta = 0,61 \lambda / (n \sin u)
\end{equation*}
По условию числовая апертура при $n = 1$ равна $0,9$, т. е. $\sin u = 0,9$. Поэтому в первом случае $\delta = 0,3 мкм$, а во втором $\delta = 0,19 мкм$.