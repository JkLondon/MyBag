\subsubsection*{7.10}
Для малых углов угловой радиус
\begin{equation}\label{eq7101}
	\theta_m  = [0,61 + (m - 1)/2] \lambda / R
\end{equation}
где $m = 1, 2, 3, \dots$\\
Для углового размера пятна имеем
\begin{equation}\label{eq7102}
	\theta \approx 1,22 \lambda / D
\end{equation}
Чтобы разрешать детали с размером $l$ на расстоянии $h$ необходимо $\theta \leq l/h$. Для диаметра объектива получаем $D \geq 1,22 \lambda h / l = 12 см$. Способность разрешения на плёнке связано с расстоянием между зёрнами плёнки, которое равно $1/N$. Детали изображения должны быть больше этого расстояния $(l/h) f \geq 1/N$, откуда $f \geq h / (Nl) = 40 см$. Размывания картины не будет , если отношение смещения объекта в системе, связанной с самолётом, к высоте полёта будет меньше отношения размера зерна плёнки (расстояние между штрихами) к факусному расстоянию фотоаппарата: $V \tau / h < (1/N) f$. Откуда
\begin{equation*}
	\tau < h / (N f V) \approx 0,25 \cdot 10^{-3} с
\end{equation*}