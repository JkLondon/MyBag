\subsubsection*{5.13}
\begin{equation}\label{eq5131}
	D \sin \theta = m \lambda
\end{equation}
Так как плоскость наблюдения находится в зоне Фраунгофера, полуширина первого максимума определяется (\ref{eq5131}), а интенсивность следующего составляет менее $4\%$ от первого, то наблюдения возможны только в пределах области на экране $2L\lambda/b$, где $L$ --- растояние от щелей до экрана. Для монохроматического источника в схеме Юнга из $\Delta y = \lambda / (2\alpha)$ получаем ширину полос $\Delta y = \lambda L / l$ и число $N = 2l /b = 40$, поскольку это число совпадает с наблюдаемым числом $N_1$, немонохроматичность ещё не уменьшает число полосб т.е. $m_{max} = \lambda / \Delta \lambda \geq 40/2$. При уменьшении $b$ в $5$ раз число полос в отсутствии немонохроматичности должно возрасти в $5$ раз. Но этого не наблюдается. И, следовательно, ограничение связано с немонохроматичностью, т.е. $m_{max} = \lambda / \Delta \lambda = 80/2 = 40$. Откуда $\Delta \lambda = \lambda / 40 = 12,5 нм$.