\subsubsection*{2.1}
Ну давайте докажем, сначала разберем случай для \textbf{E}, лежащего в плоскости падения.
\begin{equation*}
	1 - R_{\|} = T_{\|} \cos{\psi} / \cos{\varphi}, \  1 + R_{\|} = nT_{\|}
\end{equation*}
\begin{equation*}
	1 - R_{\bot} = T_{\bot} \cos{\psi} / \cos{\varphi}, \  1 + R_{\bot} = nT_{\bot}
\end{equation*}

Далее выводим, что

\begin{equation*}
	1 - R_{\|}^2 = T_{\|}^2 n\cos{\psi} / \cos{\varphi} 
\end{equation*}

\begin{equation*}
	1 - R_{\bot}^2 = T_{\bot}^2 n\cos{\psi} / \cos{\varphi} 
\end{equation*}

Для потоков энергии, в случае равенства потока энергии падающей волны потокам энергии отраженной и преломлённой волн имеем

\begin{equation*}
	\cos{\varphi} E_{e\|}^2 = \cos{\varphi} E_{r\|}^2  + \cos{\psi} E_{d\|}^2 
\end{equation*}
Вводя амплитудные коэффициенты отражения и преломления находим

\begin{equation*}
	1 = R_{\|}^2 + T_{\|}^2 n\cos{\psi} / \cos{\varphi} 
\end{equation*}

Получили из формул Френеля.\\
В случае нормального падения волны ($\varphi = \psi = 0$) находим

\begin{equation*}
	1 - R_{\|} = T_{\|}, \ 1 + R_{\|} = nT_{\|}, \ 1 + R_{\bot} = T_{\bot}, \ 1 - R_{\bot} = nT_{\bot}
\end{equation*}

Отсюда получаем 
\begin{equation*}
	R_{\bot} = -R_{\|}, \  T_{\bot} = T_{\|}
\end{equation*}
ЧТД.