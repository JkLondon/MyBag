\subsection*{11.88}
Условия резонанса: $2 n d = m \lambda$.
$$2d (n_0 + n_2 E^2) = m \lambda$$
$$E^2 = \frac{m\lambda - 2 d n_0}{2dn_2}$$
$$\vec{S} = \frac{c}{4\pi} \vec{E} \times \vec{H}$$
$$p = \frac{c \varepsilon E^2}{4 \pi} (1 - \rho) = \frac{c n_0^2}{4\pi} (1-\rho) (\frac{m\lambda - 2dn_0}{2dn_2}) = 10^{15} \frac{эрг}{см^2 \cdot с}$$
$$m_0 = \frac{2dn_0}{\lambda} = 79,92 \text{ ближаешее целое --- 80.}$$