\documentclass[a4paper,12pt]{article} % добавить leqno в [] для нумерации слева

%%% Работа с русским языком
\usepackage{cmap}					% поиск в PDF
\usepackage{mathtext} 				% русские буквы в фомулах
\usepackage[T2A]{fontenc}			% кодировка
\usepackage[utf8]{inputenc}			% кодировка исходного текста
\usepackage[english,russian]{babel}	% локализация и переносы
\usepackage{float}
\usepackage{graphicx}
%%% Дополнительная работа с математикой
\usepackage{amsmath,amsfonts,amssymb,amsthm,mathtools} % AMS
\usepackage{icomma} % "Умная" запятая: $0,2$ --- число, $0, 2$ --- перечисление

%% Номера формул
%\mathtoolsset{showonlyrefs=true} % Показывать номера только у тех формул, на которые есть \eqref{} в тексте.

%% Шрифты
\usepackage{euscript}	 % Шрифт Евклид
\usepackage{mathrsfs} % Красивый матшрифт

%% Свои команды
\DeclareMathOperator{\sgn}{\mathop{sgn}}

%% Перенос знаков в формулах (по Львовскому)
\newcommand*{\hm}[1]{#1\nobreak\discretionary{}
{\hbox{$\mathsurround=0pt #1$}}{}}

%%% Заголовок
\author{Илья Михеев}
\title{Билеты по Электричеству и Магнетизму}
\date{\today}

\begin{document} % конец преамбулы, начало документа

\maketitle 

\part{Электростатика}
\section{Электрические заряды и электрическое поле. Закон сохранения заряда, элементарный заряд. Напряжённость электрического поля. Закон Кулона. Гауссова система единиц (СГС) и система СИ. Принцип суперпозиции. Электрическое поле диполя.}
\subsection{Электрические заряды} мера взаимодействия заряженного тела с полем.
\subsection{Электрическое поле} область пространства, где действуют электрические силы.
\subsection{Закон Сохранения Заряда} экспериментальный факт, что сумма зарядов --- сохраняющаяся величина.
\subsection{Элементарный заряд} в природе заряд дискретен, его минимальная порция --- треть от \\  $e = 4,803 \cdot 10^{-10} ед. СГСЭ$
\subsection{Напряженность электрического поля} называется сила, действующая на ед. точечный заряд. Опытным путем установили, что если поместить точечный заряд $q$ в поле, то величина силы $F$, поделенная на величину заряда не зависит от этой величины.
\subsection{Закон Кулона}  Экспериментально установлено, что заряды одного знака отталкиваются, а разных --- притягиваются.
\begin{equation}
	\textbf{F} = \frac{Qq}{r^3} \textbf{r}
\end{equation}
\subsection{Гауссова система единиц (СГС) и система СИ} СГС - сантиметр, грамм, секунда, считаем разряд безразмерным. СИ --- имеем константу $k = \frac{1}{4 \pi \varepsilon_0} = 9 * 10^9$, заряд измеряем в кулонах.
\subsection{Принцип суперпозиции} $\sum_{i=1}^n E_i = const$ экспериментальный факт.
\subsection{Электрическое поле диполя}
\begin{equation}
	{\bf E  = E_{-} + E_+} = - \biggl [ l_x \frac{\partial }{\partial x} +  l_y \frac{\partial }{\partial y} +  l_z \frac{\partial }{\partial z}  \biggr ] {\bf E_0 (r)}  \equiv - {(\bf l \nabla) E_0} = \frac{3 {\bf (pr)r}  - {\bf p} r^2)}{r^5}
\end{equation}
\section{Теорема Гаусса для электрического поля в вакууме в интегральной и дифференциальной фор-мах. Её применение для нахождения электростатических полей.}
\subsection{Теорема Гаусса для электрического поля в вакууме в интегральной и дифференциальной фор-мах.}
\begin{equation}
    Ф = \oint\limits_S {\bf E} d {\bf S}
\end{equation}
\begin{equation}
	div {\bf E} = 4 \pi \rho
\end{equation}
\subsection{Доказательство Теоремы Гаусса}
\subsubsection{Точечный заряд $q$ внутри сферы радиусом $r$}
Начало кооров в центр, после $d {\bf S || r}$. Из того, что ${\bf E}  = q {\bf r} / r^3$ элементарный поток равен
\begin{equation}
	d Ф = {\bf E} d {\bf S} =  \frac{q}{r^2} d S
\end{equation}
откуда весь поток:
\begin{equation}
	Ф = \frac{q}{r^2} S = \frac{q}{r^2} 4 \pi r^2 = 4 \pi q
\end{equation}
\subsubsection{Поверхность несферическая}
\begin{equation}
	d Ф = {\bf E} d {\bf S} =  \frac{q}{r^3} {\bf r} d {\bf S} = \frac{q}{r^2} d S_{||}
\end{equation}
далее аналогично верхнему, тк. $d S_{||}$ это буквально часть сферы ну и про телесный угол сказать не забыть.
\subsubsection{Заряд вне замкнутой поверхности}
сказать, что там 2 раза протыкает заряд поверхность телесным углом (одним) с разными знаками и поэтому ноль.
\subsubsection{Заряд конченый и система зарядов}
Если заряд в конченой штуке --- все будет ок, верим. В системе зарядов говорим про суперпозицию, пока не умрем.
\subsection{Применения}
Поле равномерно заряженной плоскости: 
\begin{equation}
	Ф = 2 E d S = 4 \pi \sigma d S \Rightarrow E = 2 \pi \sigma
\end{equation}
\section{Потенциальный характер электростатического поля. Теорема о циркуляции электростатического поля. Потенциал и разность потенциалов. Связь напряжённости поля с градиентом потенциала. Граничные условия для вектора {\bf E}. }
\subsection{Потенциальный характер электростатического поля}
\begin{equation}
	A_{12} = \int\limits_{(1)}^{(2)} q {\bf E(r)} d {\bf r} = q Q  \int\limits_{(1)}^{(2)} \frac{{\bf r}d {\bf r}}{r^3} = q Q (\frac{1}{r_1} - \frac{1}{r_2})
\end{equation}
Данная шняга зависит онли от нач и кон точки $Rightarrow$ поле консервативно у точечного заряда, ну и в силу суперпозиции у их системы. Таким образом вводим потенциальную энергию заряда $q$ в этом поле: работа сил поля на пути $1 \rightarrow 2$  равна убыли потенциальной энергии заряда.
\subsection {Теорема о циркуляции электростатического поля}
Из того, что поле потенциально очевидна теорема:
\begin{equation}
	\oint\limits_L {\bf E} d {\bf r} = 0
\end{equation}
\subsection{Потенциал и разность потенциалов}
{\it Разностью потенциалов} $\varphi_1 - \varphi_2$ называют работу сил по перемещению заряда из точки 1 в точку 2. Часто за начало отсчета берут бесконечность, а сам потенциал в таком случае имеют ввиду, как функцию от {\bf r}
\subsection{Связь напряжённости поля с градиентом потенциала}
\begin{equation}
	\varphi ({\bf r}) - \varphi ({\bf r} + d  {\bf r}) = - d \varphi = {\bf E(r)} d {\bf r}
\end{equation}
откуда $E = - \nabla \varphi$.
\subsection{Граничные условия для вектора {\bf E}}
1. Очевидно из теоремы о циркуляции, что $E_{1\tau} = E_{2 \tau}$.
2. Накрываем поверхность параллелепипедом и пишем:
\begin{equation}
	\oint\limits_S {\bf E} d {\bf S} = 4 \pi q \Rightarrow E_{1n} - E_{2n} = 4 \pi \sigma
\end{equation}
\end{document} % конец документа
