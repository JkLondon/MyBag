\documentclass[a4paper,12pt]{article} % добавить leqno в [] для нумерации слева

%%% Работа с русским языком
\usepackage{cmap}					% поиск в PDF
\usepackage{mathtext} 				% русские буквы в фомулах
\usepackage[T2A]{fontenc}			% кодировка
\usepackage[utf8]{inputenc}			% кодировка исходного текста
\usepackage[english,russian]{babel}	% локализация и переносы
\usepackage{float}
\usepackage{graphicx}
%%% Дополнительная работа с математикой
\usepackage{amsmath,amsfonts,amssymb,amsthm,mathtools} % AMS
\usepackage{icomma} % "Умная" запятая: $0,2$ --- число, $0, 2$ --- перечисление

%% Номера формул
%\mathtoolsset{showonlyrefs=true} % Показывать номера только у тех формул, на которые есть \eqref{} в тексте.

%% Шрифты
\usepackage{euscript}	 % Шрифт Евклид
\usepackage{mathrsfs} % Красивый матшрифт

%% Свои команды
\DeclareMathOperator{\sgn}{\mathop{sgn}}

%% Перенос знаков в формулах (по Львовскому)
\newcommand*{\hm}[1]{#1\nobreak\discretionary{}
{\hbox{$\mathsurround=0pt #1$}}{}}

%%% Заголовок
\author{Илья Михеев}
\title{Билеты по Электричеству и Магнетизму}
\date{\today}

\begin{document} % конец преамбулы, начало документа

\maketitle 

\part{Электростатика}
\section{Электрические заряды и электрическое поле. Закон сохранения заряда, элементарный заряд. Напряжённость электрического поля. Закон Кулона. Гауссова система единиц (СГС) и система СИ. Принцип суперпозиции. Электрическое поле диполя.}
\subsection{Электрические заряды} мера взаимодействия заряженного тела с полем.
\subsection{Электрическое поле} область пространства, где действуют электрические силы.
\subsection{Закон Сохранения Заряда} экспериментальный факт, что сумма зарядов --- сохраняющаяся величина.
\subsection{Элементарный заряд} в природе заряд дискретен, его минимальная порция --- треть от \\  $e = 4,803 \cdot 10^{-10} ед. СГСЭ$
\subsection{Напряженность электрического поля} называется сила, действующая на ед. точечный заряд. Опытным путем установили, что если поместить точечный заряд $q$ в поле, то величина силы $F$, поделенная на величину заряда не зависит от этой величины.
\subsection{Закон Кулона}  Экспериментально установлено, что заряды одного знака отталкиваются, а разных --- притягиваются.
\begin{equation}
	\textbf{F} = \frac{Qq}{r^3} \textbf{r}
\end{equation}
\subsection{Гауссова система единиц (СГС) и система СИ} СГС - сантиметр, грамм, секунда, считаем разряд безразмерным. СИ --- имеем константу $k = \frac{1}{4 \pi \varepsilon_0} = 9 * 10^9$, заряд измеряем в кулонах.
\subsection{Принцип суперпозиции} $\sum_{i=1}^n E_i = const$ экспериментальный факт.
\subsection{Электрическое поле диполя}
\begin{equation}
	{\bf E  = E_{-} + E_+} = - \biggl [ l_x \frac{\partial }{\partial x} +  l_y \frac{\partial }{\partial y} +  l_z \frac{\partial }{\partial z}  \biggr ] {\bf E_0 (r)}  \equiv - {(\bf l \nabla) E_0} = \frac{3 {\bf (pr)r}  - {\bf p} r^2)}{r^5}
\end{equation}
\section{Теорема Гаусса для электрического поля в вакууме в интегральной и дифференциальной фор-мах. Её применение для нахождения электростатических полей.}
\subsection{Теорема Гаусса для электрического поля в вакууме в интегральной и дифференциальной фор-мах.}
\begin{equation}
    Ф = \oint\limits_S {\bf E} d {\bf S}
\end{equation}
\begin{equation}
	div {\bf E} = 4 \pi \rho
\end{equation}
\subsection{Доказательство Теоремы Гаусса}
\subsubsection{Точечный заряд $q$ внутри сферы радиусом $r$}
Начало кооров в центр, после $d {\bf S || r}$. Из того, что ${\bf E}  = q {\bf r} / r^3$ элементарный поток равен
\begin{equation}
	d Ф = {\bf E} d {\bf S} =  \frac{q}{r^2} d S
\end{equation}
откуда весь поток:
\begin{equation}
	Ф = \frac{q}{r^2} S = \frac{q}{r^2} 4 \pi r^2 = 4 \pi q
\end{equation}
\subsubsection{Поверхность несферическая}
\begin{equation}
	d Ф = {\bf E} d {\bf S} =  \frac{q}{r^3} {\bf r} d {\bf S} = \frac{q}{r^2} d S_{||}
\end{equation}
далее аналогично верхнему, тк. $d S_{||}$ это буквально часть сферы ну и про телесный угол сказать не забыть.
\subsubsection{Заряд вне замкнутой поверхности}
сказать, что там 2 раза протыкает заряд поверхность телесным углом (одним) с разными знаками и поэтому ноль.
\subsubsection{Заряд конченый и система зарядов}
Если заряд в конченой штуке --- все будет ок, верим. В системе зарядов говорим про суперпозицию, пока не умрем.
\subsection{Применения}
Поле равномерно заряженной плоскости: 
\begin{equation}
	Ф = 2 E d S = 4 \pi \sigma d S \Rightarrow E = 2 \pi \sigma
\end{equation}
\section{Потенциальный характер электростатического поля. Теорема о циркуляции электростатического поля. Потенциал и разность потенциалов. Связь напряжённости поля с градиентом потенциала. Граничные условия для вектора {\bf E}. }
\subsection{Потенциальный характер электростатического поля}
\begin{equation}
	A_{12} = \int\limits_{(1)}^{(2)} q {\bf E(r)} d {\bf r} = q Q  \int\limits_{(1)}^{(2)} \frac{{\bf r}d {\bf r}}{r^3} = q Q (\frac{1}{r_1} - \frac{1}{r_2})
\end{equation}
Данная шняга зависит онли от нач и кон точки $Rightarrow$ поле консервативно у точечного заряда, ну и в силу суперпозиции у их системы. Таким образом вводим потенциальную энергию заряда $q$ в этом поле: работа сил поля на пути $1 \rightarrow 2$  равна убыли потенциальной энергии заряда.
\subsection {Теорема о циркуляции электростатического поля}
Из того, что поле потенциально очевидна теорема:
\begin{equation}
	\oint\limits_L {\bf E} d {\bf r} = 0
\end{equation}
Это кстати является циркуляцией вектора $\bf E$ в интергральной форме, дифф форма теоремы:
\begin{equation}
	rot {\bf E} = 0
\end{equation}
\subsection{Потенциал и разность потенциалов}
{\it Разностью потенциалов} $\varphi_1 - \varphi_2$ называют работу сил по перемещению заряда из точки 1 в точку 2. Часто за начало отсчета берут бесконечность, а сам потенциал в таком случае имеют ввиду, как функцию от {\bf r}
\subsection{Связь напряжённости поля с градиентом потенциала}
\begin{equation}
	\varphi ({\bf r}) - \varphi ({\bf r} + d  {\bf r}) = - d \varphi = {\bf E(r)} d {\bf r}
\end{equation}
откуда $E = - \nabla \varphi$.
\subsection{Граничные условия для вектора {\bf E}}
1. Очевидно из теоремы о циркуляции, что $E_{1\tau} = E_{2 \tau}$.
2. Накрываем поверхность параллелепипедом и пишем:
\begin{equation}
	\oint\limits_S {\bf E} d {\bf S} = 4 \pi q \Rightarrow E_{1n} - E_{2n} = 4 \pi \sigma
\end{equation}
\section{Уравнения Пуассона и Лапласа. Проводники в электрическом поле. Граничные условия на поверхности проводника. Единственность решения электростатической задачи. Метод изображений. Изображение точечного заряда в проводящих плоскости и сфере}
\subsection{Уравнения Пуассона и Лапласа}
Поскольку 
\begin{equation}
	{\bf E} = - grad \varphi, div {\bf E} = 4 \pi \rho
\end{equation}
то отсюда выведем {\it уравнение Пуассона} для потенциала поля:
\begin{equation}
	div grad \varphi = - 4 \pi \rho 
\end{equation}
или $ \Delta \varphi = - 4 \pi \rho$
где введен лапласиан. Если $\rho = 0$, то получаем обычное уравнение Лапласа:
\begin{equation}
	\Delta \varphi = 0
\end{equation}
\subsection{Теорема о единственности решения}
в обоих уравнениях для единственности добавляем граничное условие: в лапласе это: $$\varphi({\bf r} \in \Gamma) = 0$$
в Пуассоне: $$\varphi({\bf r} \in \Gamma) = f({\bf r})$$
\subsubsection{Лаплас}
Говорим, что $\varphi = 0$ --- решение, теперь предположим, что есть еще, у другой функции есть место, где она не равна нулю, а во всей остальной окрестности равна. Тогда у этой функции очев будет экстремум $\Rightarrow$  сумма вторых производных будет ненулевая ЧТД.
\subsubsection{Пуассон}
Предполагаем 2 решения, потом говорим, что есть функция, равная их разности, которая удовлетворяет уравнению Лапласа, но т.к. там решение только 0, то изначальные решения равны.
\subsection{Проводники в электрическом поле}
Вещества, обладающие малым сопротивлением, в них имеются свободные заряды (элетроны), которые могут перемещаться под воздействием любых полей. В объеме проводника поле --- ноль (тк элста), откуда выходит, что все заряды на поверхности.
\subsection{Граничные условия на поверхности проводника}
Применяем в лоб теорему Гаусса и знание, что поле ноль --- получаем: $E_n = 4 \pi \sigma$ и $E_{\tau} = 0$.
Таким образом поле проводника --- Ёжик.
\subsection{Метод Изображений}
Пусть у нас есть 2 группы зарядов и эквипотенциальная поверхность, по теореме о ед. мы можем сказать, что поле в области первой группы задается однозначно или этой поверхностью, или второй группой зарядов вместе с первой.
\subsection{Изображение точечного заряда в проводящих плоскости и сфере}
\subsubsection{Плоскость}
$$q' = -q$$
\subsubsection{Сфера}
Если сфера заземлена, то заменяем ее на заряд $q' = -q R / d$ на расстоянии $b = R^2 / d$ от центра.\\
Если сфера несет заряд $q_0$, то к заряду $q'$, который расположен на том же расстоянии от центра, добавляем заряд  $$q'' = q_0 + q \frac{R}{d}$$ в центр сферы.
\section{Электрическое поле в веществе. Поляризация диэлектриков. Свободные и связанные заряды. Вектор поляризации и вектор электрической индукции. Поляризуемость частиц среды. Диэлектрическая проницаемость среды. Теорема Гаусса в диэлектриках. Граничные условия на границе двух диэлектриков.}
\subsection{Электрическое поле в веществе. Свободные и связанные заряды}
Кроме свободных зарядов есть еще {\it связанные (поляризационные)}, которые мало смещаются от своего положения равновесия под воздействием вн. сил, да еще и потом обратно возвращаются.\\
{\it Диэлектриками} называют вещество с малым количеством свободных зарядов. Они еще ток плохо проводят.
\subsection{Вектор поляризации и вектор электрической индукции}
{\it Поляризация} --- это перераспределение связанных зарядов, приводящее к появлению объемного дипольного момента среды. {\it Вектором поляризации} {\bf P} называют дипольный момент единицы объема. Вектором электрической индукции называют вектор ${\bf D = E} + 4 \pi {\bf P}$, нужен он для лучшего счета, о нем позже.
\subsection{Поляризуемость частиц среды. Диэлектрическая проницаемость среды}
Про то, что при слабых полях $${\bf P} = \alpha {\bf E}$$ откуда можно найти, что $${\bf D} = (1 + 4 \pi \alpha) {\bf E} = \varepsilon {\bf E}$$ обычно $\alpha > 0$.
\subsection{Теорема Гаусса в диэлектриках}
Представим косоугольный параллелепипед (попробуем хотя бы...), его объем $V = S l cos \theta = {\bf S l}$\\
Если на его краях расположены заряды плотностью $\sigma$, то дипольный момент фигуры ${\bf p}  = (\sigma S) {\bf l}$, тогда ${\bf P} = {\bf p} / V$ найдем его нормальную компоненту:
\begin{equation}
	P_n = \frac{\bf P S}{S} = \sigma
\end{equation}
Рассмотрим неоднородную поляризацию, для нее:
\begin{equation}
	q_{пол.} = - \oint\limits_S {\bf P} d {\bf S}
\end{equation}
откуда мгновенно получаем для вектора $D$:
\begin{equation}
	\oint\limits_{S} {\bf D} d {\bf S} = 4 \pi q_{своб}
\end{equation}
и его дифф форму
\begin{equation}
	\text{div} {\bf D} = 4 \pi \rho
\end{equation}
\subsection{Граничные условия}
Применяя всё то же самое получаем $D_{1n} - D_{2n} = 4 \pi \sigma$ и $E_{1 \tau} - E_{2 \tau} = 0$
\section{Электрическая ёмкость. Конденсаторы. Вычисление ёмкостей плоского, сферического и цилиндрического конденсаторов. Энергия электрического поля и её локализация в пространстве.Объёмная плотность энергии. Взаимная энергия зарядов. Энергия в системе заряженных проводников}
\subsection{Электрическая ёмкость}
Так вышло, что для проводника величина $q / \varphi$ не зависит от заряда  и характеризует сам проводник.
Соответственно полагают, что $C = q / \varphi$, где $C$ --- ёмкость проводника.
\subsection{Конденсаторы}
Систему из 2 проводников (заряженных) назовем конденсатором, если на одном будет заряд $q$, а на другом $-q$ и разность потенциалов между ними будет $\Delta \varphi$, то $C = q / \Delta \varphi$.
\subsection{Вычисление ёмкостей плоского, сферического и цилиндрического конденсаторов}
\subsubsection{Плоский}
$E = 4 \pi \sigma / \varepsilon$ по $2 \pi \sigma / \varepsilon$ от каждой пластины, тогда $\Delta \varphi = Ed$ откуда 
\begin{equation}
	C = \frac{q}{\Delta \varphi} = \frac{\varepsilon S}{4 \pi d}
\end{equation}
\subsubsection{Сферический}
Между его обкладками (внутри $+q$) возникает разность потенциалов $$\Delta \varphi = \frac{q}{\varepsilon R_1} - \frac{q}{\varepsilon R_2}$$ откуда $$ C = \frac{\varepsilon R_1 R_2}{R_2 - R_1}$$
\subsubsection{Цилиндрический}
Поле между обкладками находим из теоремы Гаусса:
$D = 2q / rl$, откуда $$E = \frac{2q}{\varepsilon r l}$$
откуда, интегрируя от $b$ до $a$ (внут и внеш радиусы) находим $$\varphi_+ - \varphi_- = \frac{2 q}{\varepsilon l} \ln \frac{b}{a}$$
откуда находим 
\begin{equation}
	C = \frac{\varepsilon l}{2 \ln \frac{b}{a}}
\end{equation}
\subsection{Энергия электрического поля и её локализация в пространстве}
Переносчиком взаимодействия зарядов является электрическое поле, так что оно является носителем энергии.
\subsection{Объемная плотность энергии}
$$ \delta U = \int \delta \rho ({\bf r}) \cdot \varphi({\bf r}) d V$$
из теоремы Гаусса:
$$ \delta \rho = \frac{1}{4 \pi} \text{div} \delta {\bf D}$$
$$ \delta U =  \frac{1}{4 \pi} \int  \varphi  \text{div} \delta {\bf D} d V = \frac{1}{4 \pi} \int \left[ \text{div} (\varphi \delta {\bf D}) - \delta {\bf D} \cdot \text{grad} \varphi \right] d V$$
первое слагаемое обращается в нуль при применении теоремы Остроградского-Гаусса тк на большом расстоянии поле обращается в нуль. Во втором слагаемом учтем, что $E = - \text{grad} \varphi$, тогда
$$\delta U = \int \frac{{\bf E} \delta {\bf D}}{4 \pi} d V$$
в котором всё норм при линейной зависимости. То есть при ${\bf D} = \varepsilon {\bf E}$ 
$$u = \frac{\varepsilon E^2}{8 \pi}$$
\subsection{Взаимная энергия зарядов}
$$U_{вз} = \sum\limits_{i,k i < k} \frac{q_i q_k}{r_{ik}} = \frac{1}{2} \sum\limits_{i} q_i \varphi_i$$ 
\subsection{Энергия в системе заряженных проводников}
Говорим, что допустим система из 2 зарядов создает с помощью собственных полей 2 собственные энергии, а еще одну взаимную (раскрываем скобочку)
$$U_{вз} = \int \frac{\bf E_1 E_2}{4 \pi} d V$$
эта энергия может быть и отрицательной.
\section{Энергия электрического поля в веществе. Энергия диполя во внешнем поле (жесткий и упругий диполи). Силы, действующие на диполь в неоднородном электрическом поле. Энергетиче-ский метод вычисления сил (метод виртуальных перемещений), вычисление сил при постоянных зарядах и при постоянных потенциалах}
\subsection{Энергия электрического поля в веществе}
Энергия электрического поля в веществе складывается из 2 компонент: электрической и деформации, энергия деформации молекул в объеме $$u_{деф} = \frac{\bf P E}{2}$$ складывая ее с энергией поля получим
$$u = u_{эл} + u_{деф} = \frac{\bf E D}{8 \pi}$$
\subsection{ Энергия диполя во внешнем поле (жесткий и упругий диполи)}
\subsubsection{Жесткий диполь}
на диполь действует момент сил
$$M = p E \sin \theta$$
он стремится уменьшить угол $\theta$ откуда
$$A = pE(\cos \theta - \cos \theta_0)$$
выбрав в качестве начала отсчета прямой угол тета получим:
\begin{equation}
	U = - \bf PE
\end{equation}
\end{document} % конец документа
